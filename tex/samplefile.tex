\documentclass{article}
\usepackage[portuguese]{babel}
\usepackage[utf8]{inputenc}

\begin{document}



\centerline{\sc \large A Simple foobar Sample \LaTeX\ File}
\vspace{.5pc}
\centerline{\sc Stupid Stuff I Wish Someone Had Told Me Four Years Ago}
\centerline{\it (Read the .tex file along with this or it won't 
            make much sense)}
\vspace{2pc}

The first thing to realize about \LaTeX\ is that it is not ``WYSIWYG''. 
In other words, it isn't a word processor; what you type into your 
.tex file is not what you'll see in your .dvi file.  For example, 
\LaTeX\ will      completely     ignore               extra
spaces    within                             a line of your .tex file.
Pressing return
in 
the 
middle 
of
a
line
will not register in your .dvi file. However, a double carriage-return
is read as a paragraph break.

Like this.  But any carriage-returns after the first two will be 
completely ignored; in other words, you 


can't 

add






more 




space 


between 




lines, no matter how many times you press return in your .tex file.

In order to add vertical space you have to use ``vspace''; for example, 
you could add an inch of space by typing \verb|\vspace{1in}|, like this:
\vspace{1in}

To get three lines of space you would type \verb|\vspace{3pc}|
(``pc'' stands for ``pica'', a font-relative size), like this:
\vspace{3pc}

Notice that \LaTeX\ commands are always preceeded by a backslash.  
Some commands, like \verb|\vspace|, take arguments (here, a length) in
curly brackets.  

The second important thing to notice about \LaTeX\ is that you type 
in various ``environments''...so far we've just been typing regular 
text (except for a few inescapable usages of \verb|\verb| and the
centered, smallcaps, large title).  There are basically two ways that 
you can enter and/or exit an environment;
\vspace{1pc}

\centerline{this is the first way...}

\begin{center}
	this is the second way.
\end{center}

\noindent Actually there is one more way, used above; for example, 
{\sc this way}.  The way that you get in and out of environment varies
depending on which kind of environment you want; for example, you use 
\verb|\underline| ``outside'', but \verb|\it| ``inside''; 
notice \underline{this} versus {\it this}.

The real power of \LaTeX\ (for us) is in the math environment. You 
push and pop out of the math environment by typing \verb|$|. For 
example, $2x^3 - 1 = 5$ is typed between dollar signs as
\verb|$2x^3 - 1 = 5$|. Perhaps a more interesting example is
$\lim_{N \to \infty} \sum_{k=1}^N f(t_k) \Delta t$.

You can get a fancier, display-style math 
environment by enclosing your equation with double dollar signs.  
This will center your equation, and display sub- and super-scripts in 
a more readable fashion:

$$\lim_{N \to \infty} \sum_{k=1}^N f(t_k) \Delta t.$$

If you don't want your equation to be centered, but you want the nice 
indicies and all that, you can use \verb|\displaystyle| and get your 
formula ``in-line''; using our example this is 
$\displaystyle \lim_{N \to \infty} \sum_{k=1}^N f(t_k) \Delta t.$  Of 
course this can screw up your line spacing a little bit.

There are many more things to know about \LaTeX\ and we can't 
possibly talk about them all here.
You can use \LaTeX\ to get tables, commutative diagrams, figures, 
aligned equations, cross-references, labels, matrices, and all manner 
of strange things into your documents.  You can control margins, 
spacing, alignment, {\it et cetera} to higher degrees of accuracy than 
the human eye can percieve.  You can waste entire days typesetting 
documents to be ``just so''.  In short, \LaTeX\ rules.

The best way to learn \LaTeX\ is by example. Get yourself a bunch
of .tex files, see what kind of output they produce, and figure out how
to modify them to do what you want.  There are many template and 
sample files on the department \LaTeX\ page and in real life in the 
big binder that should be in the computer lab somewhere.  Good luck!

\section{Introdução}

Tecnologias de banco de dados são o componente principal de muitos sistemas de computação. Elas permitem que a informação seja retida e compartilhada eletronicamente e a quantidade de informação armazenada nesses sistemas continua a crescer à uma taxa exponencial.

Porém, danos e uso indevido de dados afetam não apenas um único usuário ou aplicação, mas pode ter consequências desastrosas para toda a organização. A proliferação rápida de aplicações baseadas na web e sistemas de informação aumentou ainda mais o risco de exposição de bancos de dados e, assim, a proteção de dados é hoje mais crucial do que nunca.

Brechas de segurança são tipicamente categorizadas como observação não autorizada de dados, modificação incorreta de dados, e indisponibilidade de dados.  Observação não autorizada de dados resulta em revelação de informação para usuários sem dinheiro de ganhar acesso a tal informação. Todas as organizações, variando de organizações comerciais para organizações sociais, numa variedade de domínios podem sofrer grandes perdas tanto do ponto de vista financeiro como humano como consequências da observação não autorizada de dados.

Modificação incorreta de dados, tanto intencional como não intencional, resulta em um estado incorreto no banco de dados. Qualquer uso de dados incorretos pode resultar em grandes perdas para a organização. Quando a informação está indisponível, informação crucial para o funcionamento adequado da organização não está prontamente disponível quando preciso.

Assim, uma solução completa para a segurança de dados deve atender os seguintes três requisitos:

1)	Sigilo ou confidencialidade refere-se a proteção dos dados contra revelação não autorizada, 2) integridade refere-se a prevenção da modificação imprópria de dados não autorizada, e 3) disponibilidade refere-se a prevenção e recuperação do hardware e software de erros e de negações de acesso de dados maliciosas tornando o banco de dados indisponível.

Esses três requisitos surgem em praticamente todos os ambientes de aplicação.

\section{Um pouco de história}

\section{Controle de acesso}

\section{Database Audit}

As técnicas de Database Auditing tem por finalidade rastrear atividades de usuário, assim como tentativas de acesso (bem ou mal sucedidas) ao banco de dados, para assim, permitir a identificação de possíveis comprometimentos a segurança do mesmo.

Ao realizar esta prática é preciso atentar-se a certas informações chave que podem, ou não, indicar um comprometimento do seu banco. Dentre estas podemos destacar alterações indevidas na configuração do banco de dados, que podem ser um indício de que segurança foi violada. Vale lembrar que nem sempre tal alteração se deve realmente a intrusão de um indivíduo mal intencionado em seu sistema, no entanto certas mudanças podem gerar vulnerabilidade em seu banco, de modo que é preciso obter informações qualquer mudança de configuração ocorrida.

É preciso prestar especial atenção a execução de certos comandos no banco, como adição, exclusão e alteração de privilegio dos usuários (Data Control Languagem), ou  mudanças estruturais em tabelas ou alteração de tipos de atributos (Data Definition Language).

Quanto a queries DML (Data Manipulation Language), é preciso notar principalmente alterações nos dados, assim como queries ad hoc.

Todas as queries acima mencionadas são executadas de maneira legitima em determinadas situações, portanto nenhuma caracteriza um comprometimento de segurança por si mesma. Assim, é preciso verificar se elas foram executadas dentro de um contexto apropriado ou não. Por exemplo caso um funcionário de uma empresa faça uma pesquisa no banco de dados por informações consideradas sensíveis, seu comportamento sera considerado normal caso ação seja condizente com suas responsabilidades na empresa, e anormal caso contrário.

Para realizar a auditoria, pode se utilizar tanto funcionalidades disponibilizadas pelos próprios sistemas gerenciadores de banco de dados quanto soluções disponibilizados por terceiros.

Usualmente, os sistemas de gerenciamento de banco de dados vem com estas funcionalidades desativas por default, sendo necessário a que sejam ativadas pelo DBA. Geralmente as funcionalidades nativas dos gerenciadores de banco de dados se resumem a manter um log de toda a atividade ocorrida no banco, que devem ser examinado periodicamente. Existem duas principais desvantagens de utilizar os recursos nativos. A primeira é a diminuição de desempenho que o banco de dados terá devido ao custo computacional de armazenar localmente um log de todas as atividades ocorridas no banco. A segunda é a presença de um intervalo de tempo significativo entre a atividade suspeita ser registrada no log e o responsável pelo banco tomar conhecimento desta atividade. 

Soluções disponibilizadas por terceiros podem ser de 3 tipos diferentes. Podem ser baseadas em sniffar pacotes com queries SQL destinadas ao servidor de banco de dados. Podem ser baseadas apenas em software. . Ou podem ser soluções mistas, que utilizam tanto um como outro.

Ao aplicar a primeira solução, sniffar a rede em busca das queries SQL enviadas ao servidor, temos alguns problemas. Uma delas é o custo alto, pois é necessário a aquisição e manutenção de hardware dedicado a tarefa de sniffar a rede. Outro problema é que, devido a conexão estar criptografada, nem sempre é possível aplicar esta solução. Fora isso ainda é preciso considerar que, nem todas as queries seram capturadas, já que qualquer  uma que seja executada direto no servidor estará fora do alcance do sniffer.

Para se livrar do último problema e tentar garantir todas as queries sejam armazenadas, é possível incluir também um software no servidor para armazenar as queries feitas diretamente nele. No entanto, assim como as funcionalidades nativas do gerenciador, este software causaria uma perda de desempenho no servidor de banco de dados. 

A verdadeira vantagem de se ter um soluções terceirizadas ao invés de contar com as funcionalidades nativas do gerenciador de banco de dados, reside no fato de que tais soluções contam com características que as outras não possuem. Dentre estas, podemos destacar a criação de políticas, a detecção de violação de política em tempo real, mensagens de alerta e a separação de funções.

Basicamente, criação de políticas, detecção de violação de política em tempo real e as mensagens de alerta, significa que é possível criar um conjunto de regras que definam quais comportamentos são considerados normais dentro do sistema, e quais não são, que o sistema conseguira detectar em tempo real quando uma destas regras foi quebrada e que o responsável pelo banco de dados sera notificado na hora.

Separação de funções significa que, ao contrário do que acontece com o com as funções nativas do gerenciador de banco de dados, quando se lida com soluções de terceiros a responsabilidade pelo monitoramento e pela administração do banco de dados estão bem separados, de modo que é possível realizar o database auditing com a menor interferência possível do DBA.










\end{document}
